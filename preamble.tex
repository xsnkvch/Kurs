
%отступы
\usepackage[left=2cm,right=2cm,top=2cm,bottom=2cm]{geometry}

\emergencystretch=5pt 
\mathsurround=2pt   
\parindent 9.3mm   
\righthyphenmin=2

\usepackage{indentfirst} % Красная строка после заголовка
\setlength\parindent{1.25cm}

\usepackage[final]{pdfpages}

\pagestyle{headings}
\usepackage{booktabs}

\DeclareMathAlphabet\mathbfcal{OMS}{cmsy}{n}{n}

\usepackage{rotating}

\usepackage{misccorr}
\usepackage{listings}
\lstset{
%   language=Python,
    basicstyle=\footnotesize\ttfamily,
    keywordstyle=\bfseries,
    showstringspaces=false,
%    breaklines=true
}

\linespread{1.} % полуторный промежуток между строками

\usepackage{booktabs}

\usepackage{amsfonts,amssymb,eucal}
\usepackage[intlimits]{amsmath}
\usepackage{graphicx}
\usepackage{delarray}
\usepackage{longtable}
\usepackage{multirow}
\usepackage{fancyheadings}
\usepackage{pgfplots}
\pgfplotsset{width=12cm}

\usepackage{amsthm}

%Стиль страницы
\usepackage{fancyhdr}
\pagestyle{plain}

% языковая верстка при XeLatex
\usepackage{polyglossia}
\usepackage{fontspec}      %% подготавливает загрузку шрифтов Open Type, True Type и др.
\defaultfontfeatures{Ligatures={TeX},Renderer=Basic}  %% свойства шрифтов по умолчанию
\setmainfont[Ligatures={TeX,Historic}]{CMU Serif} %% задаёт основной шрифт документа


%%%% языковая верстка при PdfLatex
%\usepackage[utf8]{inputenc}
%\usepackage[russian]{babel}

\setcounter{page}{2}  % номер первой страницы журнала
