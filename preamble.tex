
%отступы
\usepackage[left=3cm,right=1cm,top=2cm,bottom=2cm]{geometry}

\emergencystretch=5pt 
\mathsurround=2pt  
\righthyphenmin=2

\setcounter{secnumdepth}{2}

\usepackage{indentfirst} % Красная строка после заголовка
\setlength\parindent{1.25cm}

\usepackage[final]{pdfpages}

\DeclareMathAlphabet\mathbfcal{OMS}{cmsy}{n}{n}

%\usepackage{misccorr}

\linespread{1.} % полуторный промежуток между строками

\usepackage{amsfonts,amssymb,eucal}
\usepackage{amsthm}
\usepackage[intlimits]{amsmath}
\usepackage{delarray}
%таблицы
\usepackage{booktabs}
\usepackage{longtable}
\usepackage{multirow}
\usepackage{rotating}

%Стиль страницы
\usepackage{fancyhdr}
\pagestyle{plain}

% языковая верстка при XeLatex
\usepackage[main=russian,english]{babel}   %% загружает пакет многоязыковой вёрстки
\usepackage{fontspec}      %% подготавливает загрузку шрифтов Open Type, True Type и др.
\defaultfontfeatures{Ligatures={TeX},Renderer=Basic}  %% свойства шрифтов по умолчанию
\setmainfont[Ligatures={TeX,Historic}]{CMU Serif} %% задаёт основной шрифт документа
\newfontfamily\cyrillicfont{CMU Serif}

%%%% языковая верстка при PdfLatex
%\usepackage[utf8]{inputenc}
%\usepackage[russian]{babel}

\newcommand{\anonsection}[1]{
    %\phantomsection % Корректный переход по ссылкам в содержании
    \section*{\centerline{\MakeUppercase{#1}}\vspace{1em}}
    %\addcontentsline{toc}{section}{#1}
}
