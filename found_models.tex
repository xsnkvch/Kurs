\subsection{Одномерные экспоненциальные семейства переходных вероятностей $\mathrm{MC}_\B$}

Для случая двоичных цепей Маркова первого порядка $p_{i0} = p_{i}, p_{i1} = 1 - p_{i},$ аналогично с $q_{ij};$ $\lambda_\ast$ и $\ev$ записаны в таблице \ref{tab:tableFIRST} в обозначениях \eqref{R_1}.
 
\begin{table}[h!]
  \begin{center}
    \caption{ }
    \label{tab:tableFIRST}
    \begin{tabular}{c|c|c|c}
     \toprule % <-- Toprule here 
       $\begin{array}{c}
      \textrm{Ограничение}\\
      \textrm{на модель}
      \end{array}$ & $\lambda_\ast$ & ${\ev}_0$ & ${\ev}_1$\\
      \toprule
      \textit{нет} & $\displaystyle \frac{1}{2} \left[ \gamma + \psi + \sqrt{(\gamma - \psi)^2 + 4 \mu \xi} \right]$ & $\displaystyle \frac{\la - \psi}{\mu} $ & 1\\
\midrule
      $\begin{array}{l}
      p_0 = p_1,\\
      q_0 = q_1
      \end{array}$ & $\gamma + \mu$ & $ \displaystyle \frac{\gamma}{\mu}$ & 1\\
\midrule
      $\begin{array}{l}
      p_0 = 1 - p_1,\\
      q_0 = 1 - q_1
      \end{array}$ & $\gamma + \mu$ & 1 & 1\\
\bottomrule
    \end{tabular}
  \end{center}
\end{table}

\subsection{Некоторые одномерные экспоненциальные семейства переходных вероятностей $\mathrm{MC}_{\B^2}$}
\label{EXP_FAM_SECOND}

В случае двоичных цепей Маркова второго порядка, сведенного к первому, $p_{ik \ k0} = p_{ik}, p_{ik \ k1} = 1 - p_{ik}, k = 0 \vee 1,$ аналогично с $q_{ik kj}.$ $\ev = ({\ev}_{00}, {\ev}_{01}, {\ev}_{10}, {\ev}_{11}).$ $\Gamma, \Delta, \Sigma, \Omega, \Psi, \Phi$ в обозначениях \eqref{R_2}.

\paragraph{1.} $p_{00} = p_{01}, p_{00} = p_{10},$ аналогично $q_{ij}$              
\begin{align*}
& \la = \frac{1}{2} \left[ \Gamma + \Omega + \sqrt{(\Gamma - \Omega)^2 + 4\Delta \Sigma}\right], \\
& \ev = \left(\frac{(\la - \Omega)(\Gamma (\la - \Omega) + \Delta\Sigma)}{\Delta^2 \lambda_\ast}, \frac{\la - \Omega}{\Delta}, \frac{\Gamma (\la - \Omega) + \Delta\Sigma}{\Delta \la}, 1 \right)
\end{align*}

\paragraph{2.} $p_{00} = p_{10}, p_{01} = p_{11},$ аналогично $q_{ij}$
\begin{equation*}
\la = \frac{1}{2} \left[\Gamma +\Omega + \sqrt{(\Gamma - \Omega)^2 +4 \Delta  \Sigma} \right], \quad \ev = \left(\frac{\Gamma (\la - \Omega)}{\Delta \Omega}, \frac{\la - \Omega}{\Omega}, \frac{\Sigma}{\Omega}, 1 \right)
\end{equation*}

\paragraph{3.} $p_{00} = 1 - p_{11}, p_{01} = 1 - p_{10},$ аналогично $q_{ij}$
\begin{align*}
\la & = \frac{1}{2} \left[\Gamma +\Phi + \sqrt{(\Gamma - \Phi)^2 +4 \Delta  \Psi} \right], \\
\ev & = \left( \frac{(\la -\Phi ) (\Phi  (\la -\Gamma )+\Delta  \Psi )}{\Delta  \la
   \Psi },\frac{\la -\Gamma }{\Psi },\frac{\Phi  (\la -\Gamma )+\Delta  \Psi
   }{\la  \Psi },1 \right)
\end{align*}

\paragraph{4.} $p_{00} = p_{01} = p_{10} = p_{11},$ аналогично $q_{ij}$
\begin{equation*}
\la = \Gamma +\Delta, \quad \ev = \left( \frac{\Gamma ^2}{\Delta ^2},\frac{\Gamma }{\Delta },\frac{\Gamma }{\Delta}, 1 \right)
\end{equation*}

\paragraph{5.} $p_{00} = 1 - p_{01} = p_{10} = p_{11},$ аналогично $q_{ij}$
\begin{equation*}
\la = \Gamma +\Delta, \quad \ev = \left( \frac{\Gamma }{\Delta }, 1, 1, 1 \right)
\end{equation*}

\paragraph{6.} $p_{00} = p_{01} = 1 - p_{10} = p_{11},$ аналогично $q_{ij}$
\begin{equation*}
\la = \Gamma +\Delta, \quad \ev = \left( \frac{\Gamma }{\Delta },\frac{\Gamma }{\Delta },\frac{\Gamma }{\Delta }, 1 \right)
\end{equation*}

\paragraph{7.} $p_{00} = p_{01} = p_{10} = 1 - p_{11},$ аналогично $q_{ij}$
\begin{equation*}
\la = \Gamma +\Delta, \quad \ev = \left( \frac{\Gamma }{\Delta }, 1, 1, 1 \right)
\end{equation*}

\paragraph{8.} $p_{00} = 1 - p_{01} = 1 - p_{10} = p_{11},$ аналогично $q_{ij}$
\begin{equation*}
\la = \Gamma +\Delta, \quad \ev = \left( 1, 1, 1, 1 \right)
\end{equation*}

\paragraph{9.} $p_{00} = 1 - p_{01} = p_{10} = 1 - p_{11},$ аналогично $q_{ij}$
\begin{equation*}
\la = \Gamma +\Delta, \quad \ev = \left( 1, \frac{\Gamma }{\Delta }, \frac{\Gamma }{\Delta }, 1 \right)
\end{equation*}

\paragraph{10.} $p_{00} = p_{01} = 1 - p_{10} = 1 - p_{11},$ аналогично $q_{ij}$
\begin{equation*}
\la = \Gamma +\Delta, \quad \ev = \left( 1, 1, 1, 1 \right)
\end{equation*}

\paragraph{11.} $p_{00} = 1 - p_{01} = 1 - p_{10} = 1 - p_{11},$ аналогично $q_{ij}$
\begin{equation*}
\la = \Gamma +\Delta, \quad \ev = \left( \frac{\Gamma }{\Delta }, \frac{\Gamma }{\Delta }, \frac{\Gamma }{\Delta }, 1 \right)
\end{equation*}