Рассмотрим вероятностное пространство $(\Omega, \mathcal{F}, \mathrm{P}).$

\begin{opr}
$n\!$-мерное семейство распределений $p(x,\theta) \!$, $\theta \in \R^{d},$ называется \textbf{экспоненциальным} c естественным параметром $\theta \!$, если существуют функции $Z(\theta), h_0 (x), h(x) = (h_1 (x), \ldots, h_d(x)) \!$, такие что, 
$$p(x,\theta) = \frac{1}{Z(\theta)}\exp \left(  h_0(x) + \theta^\prime h(x) \right). $$  
\end{opr}

Одномерное экспоненциальное семейство $r(x, \alpha) \!$, содержащее два распределения $p(x)$ и $q(x)\!$, $x \in X, |X| < \infty \!$, может быть задано в виде
\begin{equation} \label{1_DIM_EXP_FAM}
r(x, \alpha) = \frac{1}{\sum_{x \in X} p^\al (x) q^\ea (x)} p^\al (x) q^\ea (x).
\end{equation}

\begin{opr}
Последовательность случайных величин $(\xi_n)_{n \in \N},$ принимающих значения в $\mathcal{S},$ называется \textbf{цепью Маркова} с пространством состояний $\mathcal{S}$, если для любого $n \in \N$ и борелевского множества $B$ 
$$\mathrm{P} (\xi_{n + 1} \in B | \sigma(\xi_m, m \leq n)) = \mathrm{P} (\xi_{n + 1} \in B | \sigma(\xi_n)),$$
если при этом переходные вероятности $\mathrm{P}(\xi_{n+w} = a | \xi_{n} = b)$ не зависят от $n \!$, то цепь Маркова называется \textbf{однородной}.  
\end{opr}

\begin{opr}
\textbf{Двоичной} называется цепь Маркова с пространством состояний $\B = \{0, 1\}.$
\end{opr}

Матрицу переходных вероятностей однородной цепи Маркова обозначим $P = (P_{ij} = \mathrm{P}(\xi_{n+1} = i | \xi_{n} = j), i, j \in \mathcal{S}),$ вероятности $\pi(x_1) = \mathrm{P}(\xi_1 = x_1)$ будем называть вероятностями начальных состояний.

\begin{ass}
Конечномерные распределения однородной цепи Маркова имеют вид 
\begin{equation} \label{FINITE_DIST}
p(x) = \pi (x_1) \prod_{s=1}^{n-1} P_{x_s x_{s+1}}, \ x = (x_1, \ldots, x_n) \in \mathcal{S}^n,
\end{equation}
а тогда однородная цепь Маркова однозначно задается матрицей переходных вероятностей, вероятностями начальных состояний и пространством состояний.
\end{ass}

Обозначим $\MC_\mathcal{S}(\pi, P)$ цепь Маркова с пространством состояний $\mathcal{S}, $ начальным распределением $\pi(x)$ и матрицей переходных вероятностей $P.$ Множество всех цепей Маркова с пространством состояний $\mathcal{S}$ обозначим $\MC_\mathcal{S}.$

\begin{opr}
Последовательность $\left(\xi_n\right)_{n \in \N},$ называется \textbf{цепью Маркова порядка \boldmath$r$}, если для любого $n \in \N$ и борелевского $B.$ 
$$\mathrm{P} (\xi_{n + 1} \in B | \sigma(\xi_m, m \leq n)) = \mathrm{P} (\xi_{n + 1} \in B | \sigma(\xi_m, n - r + 1 \leq m \leq n))$$
\end{opr}

\begin{ass}
\label{VECTORIZE}
Пусть $\left(\xi_n\right)_{n \in \N}$ -- цепь Маркова порядка $r.$ Тогда последовательность случайных величин $\left(\eta_n = (\xi_{n+1},\ldots, \xi_{n+r-1})\right)_{n \in \N}$ со значениями из $\mathcal{S}^n$ является цепью Маркова в обычном смысле. 
\end{ass}

Введем теперь несколько понятий из теории неотрицательных матриц.

\begin{opr}
Квадратная неотрицательная матрица $A = (A_{ij})_{i,j = 1}^n$ называется \textbf{стохастической}, если
 \begin{equation}
 \label{STOCH_SUM}
 \sum_{j=1}^n A_{ij} = 1, \ i = \overline{1, n}.
 \end{equation}
\end{opr}

\begin{ass}
\label{TRANS_MATR_IS_STOCH}
Матрица переходных вероятностей цепи Маркова является стохастической.
\end{ass}

\begin{opr}
Квадратная матрица $A = (A_{ij})_{i,j = 1}^n$ называется \textbf{разложимой}, если существует разбиение множества индексов $\mathcal{I} = \{1, 2, ..., n\}$ на два не пересекающихся подмножества $ \mathcal{I}_1$ и $\mathcal{I}_2$ $(\mathcal{I}_1 \cup \mathcal{I}_2 = \mathcal{I}, \ \mathcal{I}_1 \cap \mathcal{I}_2 = \varnothing), $ такое, что $A_{ij} = 0,\  i \in \mathcal{I}_1,\ j \in \mathcal{I}_2;$ в противном случае она называется \textbf{неразложимой}.
\end{opr}

\begin{opr}
$\la$ называется собственным значением Перрона-Фробениуса матрицы $A\!$, если \\
1. $\la > 0 \!$,\\
2. $\la$ простой корень характеристического полинома $\varphi(\lambda)$ матрицы $A\!$, \\
3. любое другое собственное значение $\lambda_0$ матрицы $A$ удовлетворяет $|\lambda_0| \leq \la \!$; \\
соответствующий $\la$ собственный вектор $\ev$ называется собственным вектором Перрона-Фробениуса матрицы $A.$
\end{opr}

\begin{ttt}[Перрона-Фробениуса]
\label{T-Perr_Fro}
Неотрицательная неразложимая матрица $A$ всегда имеет собственное значение Перрона-Фробениуса $\la\!$; собственный вектор Перрона-Фробениуса $\ev$ имеет положительные координаты.
\end{ttt}

\begin{col}
\label{NEOTR_V_STOCH}
Неотрицательная $n \times n$ матрица $A$ с $\la$ и $\ev$ подобна произведению $\lambda_\ast$ и некоторой стохастической матрицы $S$, т. е. $A = \la VSV^{-1},$ где $V = \diag(\ev_1, \ldots, \ev_n).$
\end{col}
