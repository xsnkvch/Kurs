Рассмотрим две цепи Маркова $\MC_{\mathcal{S}}(\pi, P)$ и $\MC_{\mathcal{S}}(\phi, Q)$ с конечномерными распределениями $p_{n}(x)$ и $q_{n}(x), \ x \in \mathcal{S}^{n}.$ 

Попытаемся построить одномерное экспоненциальное семейство, содержащее $p_{n}(x)$ и $q_{n}(x)$ при фиксированном $n.$ 

Пусть $x = (x_1, ..., x_n) \in \mathcal{S}^n,$ т. ч. $\pi (x_1), \phi(x_1) \neq 0,$ $P_{x_s x_{s+1}}, Q_{x_s x_{s+1}} \neq 0,  s = \overline{1, n-1}.$ Используя формулы \eqref{1_DIM_EXP_FAM} и \eqref{FINITE_DIST} получим:
\begin{align}
\label{r_x_al}
&r(x, \al) = C_1 \pi^\al (x_1) \phi^\ea (x_1) \prod_{s=1}^{n-1} P_{x_s x_{s+1}}^\al Q^\ea_{x_s x_{s+1}} = C_2 \rho(x_1, \al) \prod_{s=1}^{n-1} R_{x_s x_{s+1}}, \textrm{где}\\
\label{R_to_norm}
&R_{x_s x_{s+1}} = P_{x_s x_{s+1}}^\al Q^\ea_{x_s x_{s+1}},\\
\label{rho_exp}
&\rho(x_1, \al) = \frac{\pi^\al (x_1) \phi^\ea (x_1)}{\sum_{x_1 \in \mathcal{S}} \pi^\al (x_1) \phi^\ea (x_1)}, \\
&C_1 \textrm{ и } C_2 \textrm{ функции от некоторых, пока неизвестных, переменных.} \nonumber
\end{align}

Очевидно, что \eqref{rho_exp} задает одномерное экспоненциальное семейство вероятностей начальных состояний.

Используя \eqref{R_to_norm}, составим матрицу $R = (R_{ij}, \ i, j \in \mathcal{S})\!$. Из-за невозможности деления на нуль, неопределенности, возникающей при попытке возвести нуль в нулевую степень, введем ограничение, $P$ и $Q$ имеют совпадающее множество пар индексов нулевых элементов: $P_{ij} = 0 \Leftrightarrow Q_{ij} = 0.$ Тогда:

\begin{equation}
\label{R_elements_eq}
R_{ij} = \left\lbrace
\begin{array}{lc}
0, & P_{ij} = 0, \\
P_{ij}^\alpha Q_{ij}^{1-\alpha}, &  P_{ij} \neq 0.
\end{array}
\right.
\end{equation}

Матрица $R$ неотрицательная, если она неразложимая, то по утверждению \ref{TRANS_MATR_IS_STOCH} и следствию \ref{NEOTR_V_STOCH} теоремы Перрона-Фробениуса из нее может быть получена матрица переходных вероятностей:

\begin{equation}
\label{EXP_FAM}
\mathcal{R} (\al) = \left(\mathcal{R}_{ij} = \frac{1}{\lambda_\ast} \frac{{\ev}_j}{{\ev}_i} R_{ij}, \ i,j \in \mathcal{S} \right)
\end{equation}
Тогда конечномерное распределение $\MC_{\mathcal{S}}(\rho(\al), \mathcal{R}(\al))$ имеет вид:
\begin{equation} \label{FIN_DIM_EXP}
r(x, \al) = \rho(x_1, \al) \prod_{s=1}^{n-1} \mathcal{R}_{ij} = \frac{1}{\lambda_\ast^{n-1}} \exp \left( \sum_{s=1}^{n-1} \ln Q_{x_s x_{s+1}} + \al \sum_{s=1}^{n-1} \ln \frac{P_{x_s x_{s+1}}}{Q_{x_s x_{s+1}}} + \ln \frac{\rho(x_1, \al){\ev}_n}{{\ev}_1} \right)
\end{equation}

Устремим в \eqref{FIN_DIM_EXP} $n \rightarrow \infty,$ $\ln \frac{\rho(x_1, \al){\ev}_n}{{\ev}_1} = \mathcal{O}(1),$ тогда $r(x, \al)$ --- это асимптотически экспоненциальное семейство конечномерных распределений $\MC_{\mathcal{S}}(\rho(\al), \mathcal{R}(\al)).$ 

Таким образом исследование одномерного экспоненциального семейства $\MC_{\mathcal{S}}$ сводится к исследованию семейства матриц \eqref{EXP_FAM}. Это семейство так же можно назвать экспоненциальным, что согласуется с работами [\ref{Hayasi}, \ref{Nagaoka}, \ref{Nakagawa}]. Где семейтсва называются экспоненциальными семействами матриц переходов (exponential family of transition matrices)  [\ref{Hayasi}] или экспоненциальными семействами марковских ядер (exponential family of Markov kernels) [\ref{Nagaoka}], а в случае одномерного семейтсва, $\!e\!$-геодезическими ($\!e \!$-geodesic)  [\ref{Nagaoka}, \ref{Nakagawa}]. 

Рассматриваемый случай --- однородные двоичные цепи Маркова первого и второго порядков. 

\paragraph{Однородные двоичные цепи Маркова первого порядка.}

Матрицы переходных вероятностей $P, Q \in [0, 1]^{2 \times 2}.$ Учитывая условие \eqref{STOCH_SUM}, однозначно задаются параметрами $p = (p_0, p_1),\  q = (q_0, q_1),$ будем считать, $P_{i0} = p_i, \ P_{i1} = 1 - p_i, \ i = 0, 1;$ аналогично с $Q$ и $q.$

Далее рассматриваем $p, q \in (0, 1)^{2}.$ По ним построим $R$ по \eqref{R_elements_eq}, для краткости введем следующие обозначения:
\begin{align}
\label{R_1}
& \gamma = p_0^\al q_0^\ea, & \mu = (1 - p_0)^\al (1 - q_0)^\ea, \nonumber \\
& \xi = p_1^\al q_1^\ea, & \psi = (1 - p_1)^\al (1 - q_1)^\ea. 
\end{align}

Очевидно, что построенная матрица $R$ неразложима. Тогда к ней применима теорема Перрона-Фробениуса \ref{T-Perr_Fro} и одномерное экспоненциальное семейство \eqref{EXP_FAM} может быть построено.

\paragraph{Однородные двоичные цепи Маркова второго порядка.} 

Используя утверждение \ref{VECTORIZE}, сведем рассмотрение однородные двоичные цепи Маркова второго порядка к $\MC_{\B^2}.$ Аналогично двоичным цепям Маркова первого порядка $P, Q \in [0, 1]^{4 \times 4}.$ однозначно задаются параметрами $p = (p_{00}, p_{01}, p_{10}, p_{11}), q = (q_{00}, q_{01}, q_{10}, q_{11}),$ будем считать, $P_{{\kappa}i i0} = p_{{\kappa}i}, \ P_{{\kappa}i i1} = 1 - p_{{\kappa}i}, \ i = 0, 1,$ очевидно, что $P_{{\kappa}i jl} = 0, i \neq j;$ аналогично с $Q$ и $q.$ $p, q \in (0, 1)^{4}.$

Построим $R$ по \eqref{R_elements_eq}, введя следующие обозначения:
\begin{align}
\label{R_2}
& \Gamma = p_{00}^\al q_{00}^\ea, & \Delta = (1 - p_{00})^\al(1 - q_{00})^\ea, \nonumber \\
& \Phi = p_{01}^\al q_{01}^\ea, & \Psi = (1 - p_{01})^\al(1 - q_{01})^\ea, \nonumber \\
& \Pi = p_{10}^\al q_{10}^\ea, & \Upsilon = (1 - p_{10})^\al(1 - q_{10})^\ea, \nonumber \\
& \Sigma = p_{11}^\al q_{11}^\ea, & \Theta = (1 - p_{11})^\al(1 - q_{11})^\ea. 
\end{align}

Для наглядности выпишем теперь вид матрицы $R$ в случае $\MC_{\B^2}$ в обозначениях \eqref{R_2}:
\begin{equation} \label{R_2_Matr}
R = 
\begin{pmatrix}
\Gamma & \Delta & 0 & 0 \\
0 & 0 & \Phi & \Psi \\
\Pi & \Upsilon & 0 & 0 \\
0 & 0 & \Sigma & \Theta \\
\end{pmatrix}
\end{equation}

\begin{ass}
\label{irreducable_R2}
Матрица \eqref{R_2_Matr} является неразложимой. 
\end{ass}

Утверждение \ref{irreducable_R2} легко проверяется перебором всех разбиений множества индексов матрицы \eqref{R_2_Matr} на не пересекающиеся подмножества. 

К матрице \eqref{R_2_Matr} так же применима теорема Перрона-Фробениуса \ref{T-Perr_Fro} и одномерное экспоненциальное семейтсво матриц переходных вероятностей $\mathcal{R}$ $\MC_{\B^{2}}$ может быть построено по формуле \eqref{EXP_FAM}. {\textcolor{red}{ написать про перенесение результатов на двоичные цепи Маркова второго рода}}