\subsection{Случай $\mathrm{MC}_\B$}

Характеристический многочлен матрицы $R$ в обозначениях \eqref{R_1}
\begin{equation}
 \varphi_1(\lambda) = \lambda^2 - (\gamma + \psi)\lambda + (\gamma \psi - \mu \xi).
\end{equation}

$\deg \varphi_1(\lambda) = 2,$ собственное значение Перрона-Фробениуса $\lambda_\ast$ выражается аналитически через дискриминант:

\begin{equation}
\label{L1_to_simp}
\lambda_\ast = \frac{1}{2} \left[ \gamma + \psi + \sqrt{(\gamma + \psi)^2 - 4 (\gamma \psi - \mu \xi)} \right] = \frac{1}{2} \left[ \gamma + \psi + \sqrt{(\gamma - \psi)^2 + 4 \mu \xi} \right]
\end{equation}

Найдем теперь модели, которые будут упрощать выражение \eqref{L1_to_simp}.

\begin{lem}
\label{RESHEN_1}
Пусть $a, b, c, d > 0$. Тогда
$$a^\al b^\ea = c^\al d^\ea \ \forall \alpha \in \R \Longleftrightarrow a = c \wedge b = d$$
\end{lem}

{\bf\it Доказательство.} Достаточность очевидна. 

Необходимость. Т. к. достаточность выполняется, множество решений непустое. Пусть  $\al = 0,$ тогда $b = d;$ а при $\al = 1$ $a = c.$ $\blacksquare$

\begin{lem}
\label{RESHEN_2}
Пусть $a, b, c, d \in (0, 1)$. Тогда
$$a^\al b^\ea = c^\al d^\ea \ \forall \alpha \in \R \Longleftrightarrow (1 - a)^\al (1 - b)^\ea = (1 - c)^\al (1-d)^\ea$$
\end{lem}
{\bf\it Доказательство.} $a^\al b^\ea = c^\al d^\ea \ \forall \alpha \in \R \Longleftrightarrow a = c \wedge b = d \Longleftrightarrow 1 - a = 1 - c \wedge 1-b = 1- d \Longleftrightarrow (1 - a)^\al (1 - b)^\ea = (1 - c)^\al (1-d)^\ea$ $\blacksquare$

{\bf 1.} Пусть $\gamma \psi - \mu \xi = 0 \ \forall \alpha \in \R,$ что эквивалентно
\begin{equation}
\label{L1_1EQ}
p_0^\alpha q_0^{1-\alpha} (1 - p_1)^\alpha (1 - q_1)^{1 - \alpha} =  p_1^\alpha q_1^{1-\alpha} (1 - p_0)^\alpha (1 - q_0)^{1 - \alpha} \ \forall \alpha \in \R.
\end{equation} 

По лемме \ref{RESHEN_1}, решение \eqref{L1_1EQ}: $p_0(1-p_1) = p_1(1-p_0) \wedge q_0(1-q_1) = q_1(1-q_0).$ 

$p_0 = (1 - p_0) \wedge p_1 = (1 - p_1)$ является частным случаем $p_0 = p_1$ при $p_0 = 1/2,$ аналогично и $q_0 = (1 - q_0) \wedge q_1 = (1 - q_1)$ является частным случаем $q_0 = q_1,$ поэтому далее рассматриваем решение $p_0 = p_1 \wedge q_0 = q_1.$ Упрощения дают $\lambda_\ast = \gamma + \psi.$

\begin{equation}
\lambda_\ast = p_0^\alpha q_0^{1-\alpha} + (1 - p_0)^\alpha (1 - q_0)^{1 - \alpha}, \textsl{если } p_0 = p_1 \wedge q_0 = q_1.
\end{equation}

{\bf 2.} Пусть теперь $\gamma - \psi = 0 \ \forall \alpha \in \R$. 
\begin{equation}
\label{L1_2EQ}
p_0^\alpha q_0^{1-\alpha} = (1 - p_1)^\alpha (1 - q_1)^{1-\alpha} \  \forall \alpha \in \R.
\end{equation}

Решение \eqref{L1_2EQ} $p_0 = 1 - p_1 \wedge q_0 = 1 - q_1.$
По лемме \ref{RESHEN_2} $\mu = \xi \ \forall \alpha \in \R, $ а тогда $\lambda_\ast = \gamma + \mu.$

\begin{equation}
\lambda_\ast = p_0^\alpha q_0^{1-\alpha} + (1 - p_0)^\alpha (1 - q_0)^{1-\alpha}, \textsl{если } p_0 = 1 - p_1 \wedge q_0 = 1 - q_1.
\end{equation}

\subsection{Случай $\MC_{\B^2}$}

Характеристический многочлен матрицы \eqref{R_2_Matr}: 
\begin{multline} \label{char_poly_2}
\varphi_2(\lambda) = \lambda^4 - (\Gamma + \Theta)\lambda^3 + (\Gamma \Theta -\Phi \Upsilon)\lambda^2 + (\Gamma \Phi \Upsilon + \Theta \Phi \Upsilon - \Delta \Pi \Phi - \Psi \Sigma \Upsilon)\lambda + \\ + (\Theta \Phi - \Psi \Sigma)(\Delta \Pi - \Gamma \Upsilon).
\end{multline}

Т. к. $\varphi_2 (\lambda)$ полином четвертой степени, в общем случае метод Феррари для аналитического нахождения корней очень сложен. Модели, для которых $\la$ выражается аналитически проще будем искать из вида \eqref{char_poly_2}.  

{\bf 1.} Пусть $\Gamma \Theta - \Phi \Upsilon = 0 \Longleftrightarrow p_{00} (1-p_{11}) = p_{01}(1-p_{10}) \wedge q_{00} (1-q_{11}) = q_{01}(1-q_{10}),$ для простоты сокращения будем рассматривать решения: 
 
{\bf \quad a)} $\Gamma = \Phi, \Theta = \Upsilon:$  
\begin{equation}
\label{MC2_2_1}
p_{00} = p_{01} \wedge p_{11} = p_{10} \wedge q_{00} = q_{01} \wedge q_{11} = q_{10},
\end{equation}
по лемме \ref{RESHEN_2} $\Delta = \Psi, \Pi = \Sigma.$ Преобразуем $\varphi_2(\lambda):$

\begin{multline*}
\varphi_2(\lambda) = \lambda^4 - (\Gamma + \Theta)\lambda^3 + (\Gamma \Theta - \Delta \Pi)(\Gamma + \Theta)\lambda - (\Gamma \Theta - \Delta \Pi)^2 = \\ =(\lambda^2 + \Delta \Pi - \Gamma \Theta) (\lambda^2 - (\Gamma + \Theta)\lambda + \Gamma \Theta - \Delta \Pi).
\end{multline*}

\begin{equation}
\lambda_\ast = \frac{1}{2} \left[ \Gamma + \Theta + \sqrt{(\Gamma - \Theta)^2 + 4\Delta \Pi}\right],\quad \textrm{если \eqref{MC2_2_1}}
\end{equation}

{\bf \quad b)} $\Gamma = \Upsilon, \Theta = \Phi$ не дает простого <<аналитичного>>  решения.

{\bf 2.} Пусть $\Gamma \Phi \Upsilon + \Theta \Phi \Upsilon - \Delta \Pi \Phi - \Psi \Sigma \Upsilon = 0,$ рассмотрим: 
 
{\bf 2.1.} $\Theta \Phi = \Sigma \Psi \wedge \Gamma \Upsilon = \Delta \Pi$

{\bf a)} $\Theta = \Sigma \wedge \Phi = \Psi \wedge \Gamma = \Delta \wedge \Upsilon = \Pi$\\
Вырожденный случай при $p_{ij}=q_{ij}=1/2, i,j \in {0, 1}.$ Все семейство -- одна цепь Маркова. $\lambda_\ast = 1.$

{\bf b)} $\Theta = \Sigma \wedge \Phi = \Psi \wedge \Gamma = \Pi \wedge \Upsilon = \Delta.$\\
\begin{equation}
\label{MC2_3_1}
p_{00} = p_{10} \wedge q_{00} = q_{10} \wedge p_{01}=p_{11}=q_{11}=q_{01}=1/2.
\end{equation}
Тогда $\Theta = \Sigma = \Phi = \Psi = 1/2.$

\begin{equation*}
\varphi_2(\lambda) = \lambda^4 - (\Gamma + \frac{1}{2})\lambda^3 + \frac{1}{2}(\Gamma - \Delta)\lambda^2.
\end{equation*}

Является частным случаем модели {\bf a)}.


{\bf c)} $\Theta = \Psi \wedge \Phi = \Sigma \wedge \Gamma = \Pi \wedge \Upsilon = \Delta.$\\
\begin{equation}
\label{MC2_3_3}
p_{01} = p_{11} \wedge q_{01} = q_{11} \wedge p_{00} = p_{10} \wedge q_{00} = q_{10}
\end{equation}

\begin{equation*}
\varphi_2(\lambda) = \lambda^4 - (\Gamma + \Theta)\lambda^3 + (\Gamma \Theta - \Phi \Delta)\lambda^2.
\end{equation*}

\begin{equation}
\lambda_\ast = \frac{1}{2} \left[ \Gamma + \Theta + \sqrt{(\Theta - \Gamma)^2 + 4\Phi\Delta}\right], \textsl{если \eqref{MC2_3_3}}.
\end{equation}

{\bf d)} $\Theta = \Psi \wedge \Phi = \Sigma \wedge \Gamma = \Delta \wedge \Upsilon = \Pi.$\\
\begin{equation}
\label{MC2_3_2}
p_{01} = p_{11} \wedge q_{01} = q_{11} \wedge p_{00}=p_{01}=q_{00}=q_{10}=1/2.
\end{equation}
Тогда $\Gamma = \Delta = \Pi = \Upsilon = 1/2.$

\begin{equation*}
\varphi_2(\lambda) = \lambda^4 - (\Theta + \frac{1}{2})\lambda^3 + \frac{1}{2}(\Theta - \Phi)\lambda^2.
\end{equation*}

Является частным случаем модели {\bf c)}.

{\bf 2.2.} $\Theta \Upsilon = \Delta \Pi \wedge \Gamma \Phi = \Psi \Sigma$

{\bf a)} $\Theta = \Delta \wedge \Upsilon = \Pi \wedge \Gamma = \Psi \wedge \Phi = \Sigma$

{\bf b)} $\Theta = \Pi \wedge \Upsilon = \Delta \wedge \Gamma = \Sigma \wedge \Phi = \Psi$

{\bf c)} $\Theta = \Delta \wedge \Upsilon = \Pi \wedge \Gamma = \Sigma \wedge \Phi = \Psi$

{\bf d)} $\Theta = \Pi \wedge \Upsilon = \Delta \wedge \Gamma = \Psi \wedge \Phi = \Sigma$

Модели со свойствами a) и c) не дают <<аналитичных>> сз. Модель со свойствами b) вырожденная и была описана в 2.1.a)

Рассмотрим модель со свойствами d):

\begin{equation}
\label{MC2_3_4}
p_{00} = 1 - p_{01} = p_{10} = p_{11}, \textrm{аналогично } q_{ij}
\end{equation}

Тогда $\Gamma = \Psi = \Pi = \Sigma \wedge \Delta = \Phi = \Upsilon = \Theta.$

\begin{equation*}
\varphi_2(\lambda) = \lambda^4 - (\Gamma + \Delta)\lambda^3 + (\Gamma \Delta - \Delta^2)\lambda^2 + (\Delta^3 - \Gamma^2 \Delta) = \lambda(\lambda - \Gamma - \Delta)(\lambda^2 + \Gamma \Delta - \Delta^2).
\end{equation*}

\begin{equation}
\lambda_\ast = \Gamma + \Delta, \textrm{если \eqref{MC2_3_4}}.
\end{equation}

Другие модели будем искать как похожие по структуре ограничений на $p_{ij}$ и $q_{ij}$.
Все найденные <<аналитичные>> модели с ограничениями на $p_{ij}$ и $q_{ij}$ выписаны в пункте \ref{EXP_FAM_SECOND}.
