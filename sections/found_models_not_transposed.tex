\subsection{Одномерные экспоненциальные семейства переходных вероятностей $\mathrm{MC}_\B$}

Для случая двоичных цепей Маркова первого порядка $p_{i0} = p_{i}, p_{i1} = 1 - p_{i},$ аналогично с $q_{ij};$ $\lambda_\ast$ и $\ev$ записаны в таблице \ref{tab:tableFIRST} в обозначениях \eqref{R_1}.
 
\begin{table}[h!]
  \begin{center}
    \caption{ }
    \label{tab:tableFIRST}
    \begin{tabular}{c|c|c|c}
     \toprule
       $\begin{array}{c}
      \textrm{Ограничение}\\
      \textrm{на модель}
      \end{array}$ & $\lambda_\ast$ & ${\ev}_0$ & ${\ev}_1$\\
      \toprule
      \textit{нет} & $\displaystyle \frac{1}{2} \left[ \gamma + \psi + \sqrt{(\gamma - \psi)^2 + 4 \mu \xi} \right]$ & $\displaystyle \frac{\la - \psi}{\xi} $ & 1\\
\midrule
      $\begin{array}{l}
      p_0 = p_1,\\
      q_0 = q_1
      \end{array}$ & $\gamma + \mu$ & 1 & 1\\
\midrule
      $\begin{array}{l}
      p_0 = 1 - p_1,\\
      q_0 = 1 - q_1
      \end{array}$ & $\gamma + \mu$ & 1 & 1\\
\bottomrule
    \end{tabular}
  \end{center}
\end{table}

\subsection{Некоторые одномерные экспоненциальные семейства переходных вероятностей $\mathrm{MC}_{\B^2}$}
\label{EXP_FAM_SECOND}

В случае двоичных цепей Маркова второго порядка, сведенного к первому, $p_{ik \ k0} = p_{ik}, p_{ik \ k1} = 1 - p_{ik}, k = 0 \vee 1,$ аналогично с $q_{ik kj}.$ $\ev = ({\ev}_{00}, {\ev}_{01}, {\ev}_{10}, {\ev}_{11}).$ $\Gamma, \Delta, \Sigma, \Theta, \Psi, \Phi$ в обозначениях \eqref{R_2}.

\textbf{1.} $p_{00} = p_{01}, p_{10} = p_{11},$ аналогично $q_{ij}$              
\begin{align*}
& \la = \frac{1}{2} \left[ \Gamma + \Theta + \sqrt{(\Gamma - \Theta)^2 + 4\Delta \Sigma}\right], \\
& \ev = \left(\frac{(\la - \Theta)(\Gamma (\la - \Theta) + \Delta\Sigma)}{\Sigma^2 \lambda_\ast}, \frac{\Gamma (\la - \Theta) + \Delta\Sigma}{\Sigma \la}, \frac{\la - \Theta}{\Sigma}, 1 \right)
\end{align*}

\textbf{2.} $p_{00} = p_{10}, p_{01} = p_{11},$ аналогично $q_{ij}$
\begin{equation*}
\la = \frac{1}{2} \left[\Gamma +\Theta + \sqrt{(\Gamma - \Theta)^2 +4 \Delta  \Sigma} \right], \quad \ev = \left(\frac{\la - \Theta}{\Sigma}, 1, \frac{\la - \Theta}{\Sigma}, 1 \right)
\end{equation*}

\textbf{3.} $p_{00} = 1 - p_{11}, p_{01} = 1 - p_{10},$ аналогично $q_{ij}$
\begin{equation*}
\la = \frac{1}{2} \left[\Gamma +\Phi + \sqrt{(\Gamma - \Phi)^2 +4 \Delta  \Psi} \right], \quad
\ev = (1, \frac{\Phi(\la - \Gamma) + \Delta \Psi}{\Delta \la}, \frac{\la - \Gamma}{\Delta}, 1)
\end{equation*}

\textbf{4.} $p_{00} = p_{01} = p_{10} = p_{11},$ аналогично $q_{ij}$ \\ \hspace*{1.12cm}
\textbf{5.} $p_{00} = 1 - p_{01} = p_{10} = p_{11},$ аналогично $q_{ij}$ \\ \hspace*{1.12cm}
\textbf{6.} $p_{00} = p_{01} = 1 - p_{10} = p_{11},$ аналогично $q_{ij}$ \\ \hspace*{1.12cm}
\textbf{7.} $p_{00} = p_{01} = p_{10} = 1 - p_{11},$ аналогично $q_{ij}$ \\ \hspace*{1.12cm}
\textbf{8.} $p_{00} = 1 - p_{01} = 1 - p_{10} = p_{11},$ аналогично $q_{ij}$ \\ \hspace*{1.12cm}
\textbf{9.} $p_{00} = 1 - p_{01} = p_{10} = 1 - p_{11},$ аналогично $q_{ij}$ \\ \hspace*{1.12cm}
\textbf{10.} $p_{00} = p_{01} = 1 - p_{10} = 1 - p_{11},$ аналогично $q_{ij}$ \\ \hspace*{1.12cm}
\textbf{11.} $p_{00} = 1 - p_{01} = 1 - p_{10} = 1 - p_{11},$ аналогично $q_{ij}$
\begin{equation*}
\la = \Gamma +\Delta, \quad \ev = (1, 1, 1, 1)
\end{equation*}