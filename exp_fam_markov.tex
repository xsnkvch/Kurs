Имеем две цепи Маркова, $\mathrm{MC}_\mathcal{D}(\pi, P)$ и $\mathrm{MC}_\mathcal{D}(\phi, Q),$ по которым будет строиться одномерное экспоненциальное семейство. В случае двоичных цепей Маркова первого порядка $\mathcal{D} = \B$; второго порядка, сведенного к первому, $\mathcal{D} = \B^2$.

Пусть $x = (x_1, ..., x_n) \in \mathcal{D}^n,$ т. ч. $\pi (x_1), \phi(x_1) \neq 0,$ $p_{x_s x_{s+1}}, q_{x_s x_{s+1}} \neq 0,  s = \overline{1, n-1}.$ Конечномерные распределения $\mathrm{MC}_\mathcal{D}(\pi, P)$ и $\mathrm{MC}_\mathcal{D}(\phi, Q)$ имеют вид 

\begin{align}
p(x) = \pi (x_1) \prod_{s=1}^{n-1} p_{x_s x_{s+1}},\\
q(x) = \phi (x_1) \prod_{s=1}^{n-1} q_{x_s x_{s+1}}. 
\end{align}

Построив по $P, Q$ $R_\alpha$ \eqref{R_elements_eq}, найдем cз $\lambda_\ast = \lambda_\ast (\alpha)$ и св $\ev = \ev(\al)$ Перрона-Фробениуса матрицы $R_\alpha^{\top}.$

Для получения стохастический матрицы используем лемму \ref{NEOTR_V_STOCH} и получим 

\begin{equation}
\label{EXP_FAM}
\mathcal{R} (\al) = \left(r_{ij} = \frac{1}{\lambda_\ast} \frac{{\ev}_j}{{\ev}_i} p_{ij}^\alpha q_{ij}^{1-\alpha}, \ i,j \in \mathcal{D} \right)
\end{equation}

Конечномерное распределение $\mathrm{MC}_\mathcal{D}(Z_{\pi\phi}^{-1} \pi^\al \phi^\ea, \mathcal{R}(\alpha)):$  

\begin{align}
r(x) & = \frac{1}{Z_{\pi\phi}} \pi_{x_1}^\al \phi_{x_1}^\ea  \prod_{s=1}^{n-1} r_{x_s x_{s+1}} = \frac{1}{\lambda_\ast^{n-1}} \frac{\pi_{x_1}^\al \phi_{x_1}^\ea {\ev}_{x_n}}{Z_{\pi\phi} {\ev}_{x_1}} \prod_{s=1}^{n-1} p_{x_s x_{s+1}}^\al q^\ea_{x_s x_{s+1}} = \nonumber \\ 
 & = \frac{1}{\lambda_\ast^{n-1}} \exp \left(\sum_{s=1}^{n-1} \left[ \al \ln p_{x_s x_{s+1}} + (\ea) \ln q_{x_s x_{s+1}} \right] + \ln \frac{\pi_{x_1}^\al \phi_{x_1}^\ea {\ev}_{x_n}}{Z_{\pi\phi} {\ev}_{x_1}}  \right) = \nonumber \\
 & = \frac{1}{\lambda_\ast^{n-1}} \exp \left( \left\lbrace \sum_{s=1}^{n-1} \ln q_{x_s x_{s+1}} \right\rbrace + \al \sum_{s=1}^{n-1} \left[ \ln p_{x_s x_{s+1}} - \ln q_{x_s x_{s+1}} \right] + \ln \frac{\pi_{x_1}^\al \phi_{x_1}^\ea {\ev}_{x_n}}{Z_{\pi\phi} {\ev}_{x_1}}  \right). 
\end{align}

Устремим $n \rightarrow \infty,$ $\displaystyle \ln \frac{\pi_{x_1}^\al \phi_{x_1}^\ea {\ev}_{x_n}}{Z_{\pi\phi} {\ev}_{x_1}} = \mathcal{O}(1), $ а тогда $r(x)$ -- это асимптотически экспоненциальное семейство конечномерных распределений $\mathrm{MC}_\mathcal{D}(Z_{\pi\phi}^{-1} \pi^\al \phi^\ea, \mathcal{R}(\alpha))$ с естественным параметром $\theta = \al \in \R$ и двойственным параметром $\eta = (n - 1)\frac{\mathrm{d}}{\mathrm{d} \al} \ln \lambda_\ast.$

Одномерное экспоненциальное семейство переходных вероятностей с естественным параметром $\al$ тогда будет иметь вид:

\begin{align}
r_{ij} & = \frac{1}{Z(\al)} \exp \left( h_0 (i,j) + \al h_1 (i,j) + \ell_1(j, \al) - \ell_2 (i, \al) \right),\  \textit{где}\\
& Z(\al) = \lambda_\ast\\
& h_0 (i ,j) = \ln q_{i j}\\
& h_1 (i ,j) = \ln p_{i j} - \ln q_{i j}\\
& \ell_1(j, \al) = \ln {\ev}_j\\
& \ell_2(i, \al) = \ln {\ev}_i
\end{align}

%%%%%%%%%%%%%%%%%%%%%%%%%%%%%%%%%%%%%%%%%%%%%%%%%%%%%%%%%%%%%%%%%%%%%%%%%

В случае одномерного экспоненциального семейства цепей Маркова по двум стохастическим матрицам $P = (P_{ij})_{i, j = 1}^n$ и $Q = (Q_{ij})_{i, j = 1}^n,$ с совпадающим множеством пар индексов нулевых элементов: $P_{ij} = 0 \Longleftrightarrow Q_{ij} = 0,$ строится матрица $R(\alpha) = (R_{ij}(\alpha))_{i, j = 1}^n,$ $\alpha \in \R$ по формуле \eqref{R_elements_eq}.

\begin{equation}
\label{R_elements_eq}
R_{ij} (\alpha) = \begin{cases}
0, \quad P_{ij} = 0, \\
P_{ij}^\alpha Q_{ij}^{1-\alpha},\  P_{ij} \neq 0.
\end{cases}
\end{equation}
