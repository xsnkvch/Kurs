
\section{Случай одномерного экспоненциального семейства двоичных цепей Маркова первого порядка $\mathrm{MC}_\B$}


Стохастические матрицы $P$ и $Q$ в двумерном многообразии двоичных цепей Маркова первого порядка можно однозначно определить параметрами $p = (p_0, p_1), q = (q_0, q_1).$ Будем рассматривать $p, q \in (0, 1)^2.$

\begin{equation}
\label{PQ_first}
P = \begin{pmatrix}
p_0 & 1 - p_0 \\
p_1 & 1 - p_1 \\
\end{pmatrix}, \ 
Q = \begin{pmatrix}
q_0 & 1 - q_0 \\
q_1 & 1 - q_1 \\
\end{pmatrix}. \ 
\end{equation}  

Тогда по \eqref{R_elements_eq} 
\begin{equation}
\label{R_1}
R_\al = \begin{pmatrix}
p_0^\al q_0^\ea & (1 - p_0)^\al (1 - q_0)^\ea \\
p_1^\al q_1^\ea & (1 - p_1)^\al (1 - q_1)^\ea \\
\end{pmatrix} = 
\begin{pmatrix}
\gamma & \mu \\
\xi & \psi \\
\end{pmatrix}, \quad \gamma, \mu, \xi, \psi > 0.
\end{equation}

Характеристический многочлен матрицы \eqref{R_1} $\varphi_1(\lambda) = \lambda^2 - (\gamma + \psi)\lambda + (\gamma \psi - \mu \xi).$

Очевидно, что $R_\alpha$ неразложима, по теореме \ref{T-Perr_Fro} $\lambda_\ast$ существует, а т.к. $\deg \varphi_1(\lambda) = 2,$ то $\lambda_\ast$ выражается аналитически через дискриминант:

\begin{equation}
\label{L1_to_simp}
\lambda_\ast = \frac{1}{2} \left[ \gamma + \psi + \sqrt{(\gamma + \psi)^2 - 4 (\gamma \psi - \mu \xi)} \right] = \frac{1}{2} \left[ \gamma + \psi + \sqrt{(\gamma - \psi)^2 + 4 \mu \xi} \right]
\end{equation}

Тогда в общем случае для матриц $R_\alpha,$ построенных по матрицам \eqref{PQ_first}, сз Перрона-Фробениуса выражается как:

\begin{multline}
\label{L1}
\lambda_\ast = \frac{1}{2}
\Big[ p_0^\al q_0^{1-\alpha} + (1 - p_1)^\alpha (1 - q_1)^{1 - \alpha} + \\
+ \left\lbrace (p_0^\alpha q_0^{1-\alpha} - (1 - p_1)^\alpha (1 - q_1)^{1 - \alpha})^2 + 4 p_1^\alpha q_1^{1-\alpha} (1 - p_0)^\alpha (1 - q_0)^{1 - \alpha}   \right\rbrace^\frac{1}{2} \Big]
\end{multline}
Найдем теперь модели, которые будут упрощать выражение \eqref{L1_to_simp}.

\begin{lem}
\label{RESHEN_1}
Пусть $a, b, c, d > 0$. Тогда
$$a^\al b^\ea = c^\al d^\ea \ \forall \alpha \in \R \Longleftrightarrow a = c \wedge b = d$$
\end{lem}

{\bf\it Доказательство.} Достаточность очевидна. 

Необходимость. Т. к. достаточность выполняется, множество решений непустое. Пусть  $\al = 0,$ тогда $b = d;$ а при $\al = 1$ $a = c.$ $\blacksquare$

\begin{lem}
\label{RESHEN_2}
Пусть $a, b, c, d \in (0, 1)$. Тогда
$$a^\al b^\ea = c^\al d^\ea \ \forall \alpha \in \R \Longleftrightarrow (1 - a)^\al (1 - b)^\ea = (1 - c)^\al (1-d)^\ea$$
\end{lem}
{\bf\it Доказательство.} $a^\al b^\ea = c^\al d^\ea \ \forall \alpha \in \R \Longleftrightarrow a = c \wedge b = d \Longleftrightarrow 1 - a = 1 - c \wedge 1-b = 1- d \Longleftrightarrow (1 - a)^\al (1 - b)^\ea = (1 - c)^\al (1-d)^\ea$ $\blacksquare$

{\bf 1.} Пусть $\gamma \psi - \mu \xi = 0 \ \forall \alpha \in \R,$ что эквивалентно
\begin{equation}
\label{L1_1EQ}
p_0^\alpha q_0^{1-\alpha} (1 - p_1)^\alpha (1 - q_1)^{1 - \alpha} =  p_1^\alpha q_1^{1-\alpha} (1 - p_0)^\alpha (1 - q_0)^{1 - \alpha} \ \forall \alpha \in \R.
\end{equation} 

По лемме \ref{RESHEN_1}, решение \eqref{L1_1EQ}: $p_0(1-p_1) = p_1(1-p_0) \wedge q_0(1-q_1) = q_1(1-q_0).$ 

$p_0 = (1 - p_0) \wedge p_1 = (1 - p_1)$ является частным случаем $p_0 = p_1$ при $p_0 = 1/2,$ аналогично и $q_0 = (1 - q_0) \wedge q_1 = (1 - q_1)$ является частным случаем $q_0 = q_1,$ поэтому далее рассматриваем решение $p_0 = p_1 \wedge q_0 = q_1.$ Упрощения дают $\lambda_\ast = \gamma + \psi.$

\begin{equation}
\lambda_\ast = p_0^\alpha q_0^{1-\alpha} + (1 - p_0)^\alpha (1 - q_0)^{1 - \alpha}, \textsl{если } p_0 = p_1 \wedge q_0 = q_1.
\end{equation}

{\bf 2.} Пусть теперь $\gamma - \psi = 0 \ \forall \alpha \in \R$. 
\begin{equation}
\label{L1_2EQ}
p_0^\alpha q_0^{1-\alpha} = (1 - p_1)^\alpha (1 - q_1)^{1-\alpha} \  \forall \alpha \in \R.
\end{equation}

Решение \eqref{L1_2EQ} $p_0 = 1 - p_1 \wedge q_0 = 1 - q_1.$
По лемме \ref{RESHEN_2} $\mu = \xi \ \forall \alpha \in \R, $ а тогда $\lambda_\ast = \gamma + \mu.$

\begin{equation}
\lambda_\ast = p_0^\alpha q_0^{1-\alpha} + (1 - p_0)^\alpha (1 - q_0)^{1-\alpha}, \textsl{если } p_0 = 1 - p_1 \wedge q_0 = 1 - q_1.
\end{equation}

\section{Случай одномерного экспоненциального семейства двоичных цепей Маркова второго порядка $MC_{\B^2}$}

Аналогично случаю $\mathrm{MC}_\B,$  $P$ и $Q$ будем определять параметрами $p = (p_{00}, p_{01}, p_{10}, p_{11}), q = (q_{00}, q_{01}, q_{10}, q_{11}).$ Будем рассматривать $p, q \in (0, 1)^4.$


\begin{equation}
\label{PQ_first}
P = \begin{pmatrix}
p_{00} & 1 - p_{00} & 0 & 0 \\
0 & 0 & p_{01} & 1 - p_{01} \\
p_{10} & 1 - p_{10} & 0 & 0 \\
0 & 0 & p_{11} & 1 - p_{11} \\
\end{pmatrix}, \ 
Q = \begin{pmatrix}
q_{00} & 1 - q_{00} & 0 & 0 \\
0 & 0 & q_{01} & 1 - q_{01} \\
q_{10} & 1 - q_{10} & 0 & 0 \\
0 & 0 & q_{11} & 1 - q_{11} \\
\end{pmatrix}. \ 
\end{equation}  

Тогда по \eqref{R_elements_eq} 
\begin{align}
\label{R_2}
& R_\alpha = 
\begin{pmatrix}
\Gamma & \Delta & 0 & 0 \\
0 & 0 & \Phi & \Psi \\
\Pi & \Upsilon & 0 & 0 \\
0 & 0 & \Sigma & \Omega \\
\end{pmatrix}, \\
& \Gamma = p_{00}^\al q_{00}^\ea, \quad \Delta = (1 - p_{00})^\al(1 - q_{00})^\ea, \nonumber \\
& \Phi = p_{01}^\al q_{01}^\ea, \quad \Psi = (1 - p_{01})^\al(1 - q_{01})^\ea, \nonumber \\
& \Pi = p_{10}^\al q_{10}^\ea, \quad \Upsilon = (1 - p_{10})^\al(1 - q_{10})^\ea, \nonumber \\
& \Sigma = p_{11}^\al q_{11}^\ea, \quad \Omega = (1 - p_{11})^\al(1 - q_{11})^\ea. \nonumber
\end{align} 

Характеристический многочлен матрицы \eqref{R_2}: 
\begin{multline}
\varphi_2(\lambda) = \lambda^4 - (\Gamma + \Omega)\lambda^3 + (\Gamma \Omega -\Phi \Upsilon)\lambda^2 + (\Gamma \Phi \Upsilon + \Omega \Phi \Upsilon - \Delta \Pi \Phi - \Psi \Sigma \Upsilon)\lambda + \\ + (\Omega \Phi - \Psi \Sigma)(\Delta \Pi - \Gamma \Upsilon).
\end{multline}

\begin{lem}
Матрица \eqref{R_2} является неразложимой. 
\end{lem}
{\bf\it Доказательство.}  Доказывается простым перебором. $\blacksquare$

\vspace{1em}

{\bf 1.} Пусть $\Gamma \Omega - \Phi \Upsilon = 0 \Longleftrightarrow p_{00} (1-p_{11}) = p_{01}(1-p_{10}) \wedge q_{00} (1-q_{11}) = q_{01}(1-q_{10}),$ для простоты сокращения будем рассматривать решения: 
 
{\bf \quad a)} $\Gamma = \Phi, \Omega = \Upsilon:$  
\begin{equation}
\label{MC2_2_1}
p_{00} = p_{01} \wedge p_{11} = p_{10} \wedge q_{00} = q_{01} \wedge q_{11} = q_{10},
\end{equation}
по лемме \ref{RESHEN_2} $\Delta = \Psi, \Pi = \Sigma.$ Преобразуем $\varphi_2(\lambda):$

\begin{multline*}
\varphi_2(\lambda) = \lambda^4 - (\Gamma + \Omega)\lambda^3 + (\Gamma \Omega - \Delta \Pi)(\Gamma + \Omega)\lambda - (\Gamma \Omega - \Delta \Pi)^2 = \\ =(\lambda^2 + \Delta \Pi - \Gamma \Omega) (\lambda^2 - (\Gamma + \Omega)\lambda + \Gamma \Omega - \Delta \Pi).
\end{multline*}

\begin{equation}
\lambda_\ast = \frac{1}{2} \left[ \Gamma + \Omega + \sqrt{(\Gamma - \Omega)^2 + 4\Delta \Pi}\right],\quad \textsl{если \eqref{MC2_2_1}}
\end{equation}

{\bf \quad b)} $\Gamma = \Upsilon, \Omega = \Phi$ не дает простого <<аналитичного>>  решения.

{\bf 2.} Пусть $\Gamma \Phi \Upsilon + \Omega \Phi \Upsilon - \Delta \Pi \Phi - \Psi \Sigma \Upsilon = 0,$ рассмотрим: 
 
{\bf 2.1.} $\Omega \Phi = \Sigma \Psi \wedge \Gamma \Upsilon = \Delta \Pi$

{\bf a)} $\Omega = \Sigma \wedge \Phi = \Psi \wedge \Gamma = \Delta \wedge \Upsilon = \Pi$\\
Вырожденный случай при $p_{ij}=q_{ij}=1/2, i,j \in {0, 1}.$ Все семейство -- одна цепь Маркова. $\lambda_\ast = 1.$

{\bf b)} $\Omega = \Sigma \wedge \Phi = \Psi \wedge \Gamma = \Pi \wedge \Upsilon = \Delta.$\\
\begin{equation}
\label{MC2_3_1}
p_{00} = p_{10} \wedge q_{00} = q_{10} \wedge p_{01}=p_{11}=q_{11}=q_{01}=1/2.
\end{equation}
Тогда $\Omega = \Sigma = \Phi = \Psi = 1/2.$

\begin{equation*}
\varphi_2(\lambda) = \lambda^4 - (\Gamma + \frac{1}{2})\lambda^3 + \frac{1}{2}(\Gamma - \Delta)\lambda^2.
\end{equation*}

Является частным случаем модели {\bf a)}.


{\bf c)} $\Omega = \Psi \wedge \Phi = \Sigma \wedge \Gamma = \Pi \wedge \Upsilon = \Delta.$\\
\begin{equation}
\label{MC2_3_3}
p_{01} = p_{11} \wedge q_{01} = q_{11} \wedge p_{00} = p_{10} \wedge q_{00} = q_{10}
\end{equation}

\begin{equation*}
\varphi_2(\lambda) = \lambda^4 - (\Gamma + \Omega)\lambda^3 + (\Gamma \Omega - \Phi \Delta)\lambda^2.
\end{equation*}

\begin{equation}
\lambda_\ast = \frac{1}{2} \left[ \Gamma + \Omega + \sqrt{(\Omega - \Gamma)^2 + 4\Phi\Delta}\right], \textsl{если \eqref{MC2_3_3}}.
\end{equation}

{\bf d)} $\Omega = \Psi \wedge \Phi = \Sigma \wedge \Gamma = \Delta \wedge \Upsilon = \Pi.$\\
\begin{equation}
\label{MC2_3_2}
p_{01} = p_{11} \wedge q_{01} = q_{11} \wedge p_{00}=p_{01}=q_{00}=q_{10}=1/2.
\end{equation}
Тогда $\Gamma = \Delta = \Pi = \Upsilon = 1/2.$

\begin{equation*}
\varphi_2(\lambda) = \lambda^4 - (\Omega + \frac{1}{2})\lambda^3 + \frac{1}{2}(\Omega - \Phi)\lambda^2.
\end{equation*}

Является частным случаем модели {\bf c)}.

{\bf 2.2.} $\Omega \Upsilon = \Delta \Pi \wedge \Gamma \Phi = \Psi \Sigma$

{\bf a)} $\Omega = \Delta \wedge \Upsilon = \Pi \wedge \Gamma = \Psi \wedge \Phi = \Sigma$

{\bf b)} $\Omega = \Pi \wedge \Upsilon = \Delta \wedge \Gamma = \Sigma \wedge \Phi = \Psi$

{\bf c)} $\Omega = \Delta \wedge \Upsilon = \Pi \wedge \Gamma = \Sigma \wedge \Phi = \Psi$

{\bf d)} $\Omega = \Pi \wedge \Upsilon = \Delta \wedge \Gamma = \Psi \wedge \Phi = \Sigma$

Модели со свойствами a) и c) не дают <<аналитичных>> сз. Модель со свойствами b) вырожденная и была описана в 2.1.a)

Рассмотрим модель со свойствами d):

\begin{equation}
\label{MC2_3_4}
p_{00} = 1 - p_{01} = p_{10} = p_{11}, \textsl{аналогично } q_{ij}
\end{equation}

Тогда $\Gamma = \Psi = \Pi = \Sigma \wedge \Delta = \Phi = \Upsilon = \Omega.$

\begin{equation*}
\varphi_2(\lambda) = \lambda^4 - (\Gamma + \Delta)\lambda^3 + (\Gamma \Delta - \Delta^2)\lambda^2 + (\Delta^3 - \Gamma^2 \Delta) = \lambda(\lambda - \Gamma - \Delta)(\lambda^2 + \Gamma \Delta - \Delta^2).
\end{equation*}

\begin{equation}
\lambda_\ast = \Gamma + \Delta, \textsl{если \eqref{MC2_3_4}}.
\end{equation}

Другие модели будем искать как похожие по структуре ограничений на $p_{ij}$ и $q_{ij}$.
Все найденные <<аналитичные>> модели с ограничениями на $p_{ij}$ и $q_{ij}$ выписаны в пункте \ref{EXP_FAM_SECOND}.