
%%%%%%%%%% Перенос формул на следующую строку (дублирование знаков) %%%%%%
\mathcode`\+="8000 {\catcode`\+=13
\global\def+{\mathchar"202B\discretionary{}{\rm\char'53}{}}
%\global\def+{\discretionary{\rm\char'53}{}{}\mathchar"202B}
}

\mathcode`\=="8000 {\catcode`\==13
\global\def={\mathchar"303D\discretionary{}{\rm\char"3D}{}} }

\mathcode`\-="8000 {\catcode`\-=13
\global\def-{\mathchar"2200\discretionary{}{\the\textfont2\char"00}{}} }

\mathcode`\<="8000 {\catcode`\<=13
\global\def<{\mathchar"313C\discretionary{}{\the\textfont1\char"3C}{}} }

\mathcode`\>="8000 {\catcode`\>=13
\global\def>{\mathchar"313E\discretionary{}{\the\textfont1\char"3E}{}} }

\global\def\subset{\mathchar"321A\discretionary{}{\the\textfont2\char"1A}{}}
\global\def\supset{\mathchar"321B\discretionary{}{\the\textfont2\char"1B}{}}
\global\def\in{\mathchar"3232\discretionary{}{\the\textfont2\char"32}{}}
\global\def\ni{\mathchar"3233\discretionary{}{\the\textfont2\char"33}{}}
\global\def\star{\mathchar"213F\discretionary{}{\the\textfont1\char"3F}{}}
\global\def\equiv{\mathchar"3211\discretionary{}{\the\textfont2\char"11}{}}
\global\def\ge{\mathchar"363E\discretionary{}{\the\textfont6\char"3E}{}}
\global\def\le{\mathchar"3636\discretionary{}{\the\textfont6\char"36}{}}
\global\def\cup{\mathchar"225B\discretionary{}{\the\textfont2\char"5B}{}}
\global\def\cap{\mathchar"225C\discretionary{}{\the\textfont2\char"5C}{}}
\global\def\to{\mathchar"3221\discretionary{}{\the\textfont2\char"21}{}}
%\global\def\toto{\mathchar"3613\discretionary{}{\the\textfont6\char"13}{}}
\global\def\bo{\mathchar"363C\discretionary{}{\the\textfont6\char"3C}{}}
\global\def\me{\mathchar"3634\discretionary{}{\the\textfont6\char"34}{}}
\global\def\pm{\mathchar"2206\discretionary{}{\the\textfont2\char"06}{}}
\global\def\mp{\mathchar"2207\discretionary{}{\the\textfont2\char"07}{}}
\global\def\times{\mathchar"2202\discretionary{}{\the\textfont2\char"02}{}}


\global\def\defeq{:=\discretionary{}{\rm:=}{}}
\global\def\eqdef{=\colon\discretionary{}{\rm=:}{}}
\global\def\ne{\not=\discretionary{}{\the\textfont2\char"36\rm=}{}}
\global\def\noin{\notin\discretionary{}
{\textrm{$\notin$}\kern-2pt}{}}

%%%%%%%%%%%%%%%%%%%%%%%%%%%% Конец переноса формул %%%%%%%%%%%%%%%%%%%%5

%\def \No {${\cal N}^{\underline{\circ}}\!$}
%\newfont{\wncyr}{wncyr10 at 14,4 pt}  % шрифт для номера, табл. 75
%\newcommand \No {{\wncyr \symbol{125}}}

\def \kk {\varkappa}
%{\textrm{\bf c}} %{\textrm{\bf \msbm \symbol{123}}}  % kappa

\def \su {\\[+0.3ex]}  % строка перед низкой формулой
\def \sv {\\[+1.3ex]}  % строка перед высокой формулой
\def \sn {\\[-0.7ex]}  % обычная строка
\def \sk {\\[+0.4ex]}  % строка после высокой формулы
\def \sa {\\[-4ex]}    % строка перед новым абзацем
\def \sd {\\[-4.5ex]}  % строка между низкими формулами
\def \sh {\\[-1.6ex]}  % строка между высокими формулами
\def \sm {\\[-2.5ex]}  % строка между низкой и~высокой формулами

%\lineskip 1pt            % \lineskip is 1pt for all font sizes.
%\normallineskip 1pt
%\def\baselinestretch{1}

\newfont{\msbm}{msbm10}  % шрифт ажурный
%\newcommand \C {\textrm{\msbm \symbol{67}}}
\newcommand \N {\textrm{\msbm \symbol{78}}}
\newcommand \B {\textrm{\msbm \symbol{66}}}
\newcommand \aO {\textrm{\msbm \symbol{79}}}
\newcommand \R {\textrm{\msbm \symbol{82}}}
\newcommand \Z {\textrm{\msbm \symbol{90}}}
%\newcommand \k {\textrm{\msbm \symbol{123}}}  % kappa
\newfont{\msam}{msam10}  % дополнительные знаки, стрелки, табл. 71
\newcommand \toto {\textrm{\msam \symbol{19}}}% равномерное стремление

\newfont{\eusmfont}{eusm10}  % шрифт рукописный, табл. 65а
\newfont{\eusmifont}{eusm8}  % шрифт рукописный индексный, табл. 65а

\newcommand{\tr}{\mathop{\rm tr}\nolimits}
\newcommand{\const}{\mathop{\rm const}\nolimits}
\newcommand{\diag}{\mathop{\rm diag}\nolimits}
\newcommand{\dist}{\mathop{\rm dist}\nolimits}
\newcommand{\grad}{\mathop{\rm grad}\nolimits}
\newcommand{\rank}{\mathop{\rm rank}\nolimits}
\newcommand{\rang}{\mathop{\rm rang}\nolimits}
\newcommand{\rot}{\mathop{\rm rot}\nolimits}
\newcommand{\sign}{\mathop{\rm sign}\nolimits}
\newcommand{\sgn}{\mathop{\rm sgn}\nolimits}
\newcommand{\supp}{\mathop{\rm supp}\nolimits}


%%%%%%%%%%%%%%%%%%%%%%%%%%%%%%%% отечественные знаки неравенств малые                                                                         %
\newcommand \ile {\mathbin{\raisebox{0,2mm}[-0,2mm]{\msami \symbol{54}}}}%             %
\newcommand \ige {\mathbin{\raisebox{0,2mm}[-0,2mm]{\msami \symbol{62}}}}%             %
                                                                        %
%%%%%%%%%%%%%%%%%%%%%%%%%%%%%%%%     отечественные знаки неравенств
                                                                   %
\renewcommand \le {\mathbin{\textrm{\msam \symbol{54}}}}             %
\renewcommand \ge {\mathbin{\textrm{\msam \symbol{62}}}}             %
\renewcommand \leq {\mathbin{\textrm{\msam \symbol{54}}}}            %
\renewcommand \geq {\mathbin{\textrm{\msam \symbol{62}}}}            %
%%%%%%%%%%%%%%%%%%%%%%%%%%%%%%%%%%%%%%%%%%%%%%%%%%%%%%%%%%%%%%%%%%%%

\sloppy

%
% ЯЗЫК РУССКИЙ
%
% МАТЕМАТИЧЕСКИЕ ОБЪЕКТЫ
%
\language=1
\newtheorem{ttt}{\hspace{\parindent}Теорема}[part]
\newtheorem{col}{\hspace{\parindent}Следствие}[section]
\newtheorem{lem}{\hspace{\parindent}Лемма}[part]
\newtheorem{ass}{\hspace{\parindent}Утверждение}[part]
\newtheorem{rem}{\hspace{\parindent}Замечание}[section]
\newtheorem{prim}{\hspace{\parindent}Пример}[section]
\newtheorem{prp}{\hspace{\parindent}Предложение}[section]
\newtheorem{opr}{\hspace{\parindent}Определение}[section]


\renewcommand{\thelem}{\arabic{lem}}
\renewcommand{\theass}{\arabic{ass}}
\renewcommand{\therem}{\arabic{rem}}
\renewcommand{\theprp}{\arabic{prp}}
\renewcommand{\thettt}{\arabic{ttt}}
\renewcommand{\thecol}{\arabic{col}}
\renewcommand{\theprim}{\arabic{prim}}
\renewcommand{\theopr}{\arabic{opr}}
