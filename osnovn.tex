Рассмотрим вероятностное пространство $(\Omega, \mathcal{F}, \mathrm{P})$
и заданную на нем последовательность случайных величин $(\xi_n)_{n \in \N},$ принимающих значения в $\mathcal{D}.$

Пусть $\mathcal{F}_n = \sigma(\xi_m, m \leq n)$ и $\mathcal{G}_n = \sigma(\xi_n).$

\begin{opr}
Последовательность $\left(\xi_n\right)_{n \in \N},$ называется \textbf{цепью Маркова}, если для любого $n \in \N$ $\sigma\!$-алгебры $\mathcal{F}_n$ и $\mathcal{G}_n$ условно независимы отностительно $\xi_n,$ т.е. для любых борелевских $A \in \mathcal{F}_n, B \in \mathcal{G}_n$ 
$$\mathrm{P} (A \cap B | \xi_n) = \mathrm{P} (A | \xi_n) \mathrm{P} (B | \xi_n) \ \textrm{п. н.}$$
\end{opr}

\begin{opr}
Последовательность $\left(\xi_n\right)_{n \in \N},$ называется \textbf{цепью Маркова порядка \boldmath$r$}, если для любого $n \in \N$ и борелевского $B.$ 
$$\mathrm{P} (\xi_{n + 1} \in B | \sigma(\xi_m, m \leq n)) = (\xi_{n + 1} \in B | \sigma(\xi_m, n - r + 1 \leq m \leq n))$$
\end{opr}

\begin{lem}
Пусть $\left(\xi_n\right)_{n \in \N}$ -- цепь Маркова порядка $r.$ Тогда последовательность случайных величин $\left(\eta_n = (\xi_{n+1},\ldots, \xi_{n+r-1})\right)_{n \in \N} \!$, со значениями из $\mathcal{D}^n$ является цепью Маркова в обычном смысле. 
\end{lem}

$\MC_\mathcal{D}(\pi, P)$ -- цепь Маркова с множеством состояний $\mathcal{D}, $ начальным распределением $\pi(i), \ i \in \mathcal{D}$ и матрицей переходных вероятностей $P = (p_{ij}, \ i, j \in \mathcal{D}).$

В случае одномерного экспоненциального семейства цепей Маркова по двум стохастическим матрицам $P = (P_{ij})_{i, j = 1}^n$ и $Q = (Q_{ij})_{i, j = 1}^n,$ с совпадающим множеством пар индексов нулевых элементов: $P_{ij} = 0 \Longleftrightarrow Q_{ij} = 0,$ строится матрица $R(\alpha) = (R_{ij}(\alpha))_{i, j = 1}^n,$ $\alpha \in \R$ по формуле \eqref{R_elements_eq}.

\begin{equation}
\label{R_elements_eq}
R_{ij} (\alpha) = \begin{cases}
0, \quad P_{ij} = 0, \\
P_{ij}^\alpha Q_{ij}^{1-\alpha},\  P_{ij} \neq 0.
\end{cases}
\end{equation}

\section*{Теоремы и определения из теории неотрицательных матриц}

\begin{opr}
Квадратная неотрицательная матрица $A = (A_{ij})_{i,j = 1}^n$ называется \textbf{стохастической}, если
 $\sum_{j=1}^n A_{ij} = 1, \ i = \overline{1, n}.$
\end{opr}

\begin{opr}
Квадратная матрица $A = (A_{ij})_{i,j = 1}^n$ называется \textbf{разложимой}, если существует разбиение множества индексов $\mathcal{I} = \{1, 2, ..., n\}$ на два непересекающихся подмножества $ \mathcal{I}_1$ и $\mathcal{I}_2$ $(\mathcal{I}_1 \cup \mathcal{I}_2 = \mathcal{I}, \ \mathcal{I}_1 \cap \mathcal{I}_2 = \varnothing), $ такое, что $A_{ij} = 0,\  i \in \mathcal{I}_1,\ j \in \mathcal{I}_2;$ в противном случае она называется \textbf{неразложимой}.
\end{opr}

\begin{opr}
$\la$ называется собсвенным значением Перрона-Фробениуса матрицы $A\!$, если \\
1. $\la > 0 \!$,\\
2. $\la$ простой корень характеристического полинома $\varphi(\lambda)$ матрицы $A\!$, \\
3. любое другое собственное значение $\lambda_0$ матрицы $A$ удовлетворяет $|\lambda_0| \leq \la \!$, \\
а соответствующий $\la$ собтвенный вектор $\ev$ называется собственным вектором Перрона-Фробениуса матрицы $A.$
\end{opr}

\begin{ttt}[Перрона-Фробениуса]
\label{T-Perr_Fro}
Неотрицательная неразложимая матрица $A$ всегда имеет собсвенное значение Перрона-Фробениуса $\la\!$; собственный вектор Перрона-Фробениуса $\ev$ имеет положительные координаты.
\end{ttt}

\begin{col}
\label{NEOTR_V_STOCH}
Неотрицательная $n \times n$ матрица $A$ с $\la$ и $\ev$ подобна произведению $\lambda_\ast$ и некоторой стохастической матрицы $S$, т. е. $A = \la VSV^{-1},$ где $V = \diag(\ev_1, \ldots, \ev_n).$
\end{col}